\chapter{Pendahuluan}

\section{Latar Belakang}

Adanya pensintesis suara alat musik memungkinan musik dimainkan dari partitur dengan biaya lebih rendah, tanpa perlu melibatkan pemain manusia. Permainan ini dapat digunakan baik untuk kakas komposisi, memainkan musik yang dibangkitkan komputer, maupun menyediakan iringan bagi musisi manusia. Dalam mensintesis suara alat musik, terdapat dua pendekatan yaitu pendekatan yang tidak ekspresif dan pendekatan ekspresif. Dengan pendekatan non-ekspresif, pensintesis suara memainkan musik dari partitur dengan waktu yang tepat metronomik, serta hanya memainkan sesuai dengan partitur. Pendekatan lainnya adalah pendekatan ekspresif, yaitu seperti permainan manusia, memberikan perubahan atau tambahan interpretasi yang tidak tertulis dalam partitur, misalnya perubahan tempo, dinamika, dan artikulasi. Secara lebih umum, sistem komputer untuk menghasilkan suara ekspresif, baik dengan sintesis atau dengan alat musik akustik yang dimainkan oleh kontroler yang terhubung komputer, disebut sebagai sistem komputer untuk permainan musik ekspresif (CSEMP). Meski permainan dan sintesis non-ekspresif lebih populer untuk ragam synth-pop dan dance music, untuk genre lain seperti musik klasik permainan yang non-ekspresif kurang disukai. Untuk genre seperti ini, pendengar dan penerbit musik lebih memilih permainan ekspresif, baik permainan musisi manusia atau sintesis dengan taraf kealamian yang tinggi seperti musisi manusia.\parencite{Kirke:2009:SCS:1592451.1592454}

Untuk beberapa alat musik, telah dikembangkan sistem untuk permainan musik ekspresif yang terotomasi sepenuhnya, menerima partitur dan menghasilkan suara ekspresif. Dalam riset \citet{schubert2017test}, dirangkumkan beberapa CSEMP untuk piano yang mampu menerima masukan partitur non-ekspresif dan menghasilkan suara piano, dengan bantuan piano akustik berkontroler. Permainan beberapa CSEMP untuk piano dalam penelitian \citet{schubert2017test} telah terbukti setara dengan pemain manusia oleh pendengar manusia, dari sisi ukuran kealamian maupun preferensi pendengar. Meski hasil tersebut diperoleh dalam berbagai batasan, riset tersebut menunjukkan bahwa CSEMP dapat mencapai taraf kealamian dan preferensi yang tinggi. Sistem lain yang diberikan dalam riset \citet{bonada2017singing} untuk suara vokal juga mampu menerima partitur non-ekspresif dan menghasilkan suara ekspresif, dengan peningkatan kealamian dari sistem-sistem sebelumnya.

Belum ada sistem yang mampu menghasilkan suara alat musik gesek ekspresif dari partitur non-ekspresif saja. Riset-riset yang ada terkait CSEMP alat musik gesek baru berupa riset terhadap komponen-komponen dari CSEMP \parencite{marchini2014quartet} \parencite{yu2017bowing} \parencite{lindemann2007rpm} \parencite{yang2016synthesis} \parencite{nsynth2017}. Namun, komponen-komponen tersebut tidak dapat langsung digabungkan karena output dari satu komponen tidak dapat langsung dijadikan input komponen lain.

Di antara riset terkait komponen CSEMP alat musik gesek hanya memberikan komponen perencana gestur ekspresif. Dalam penelitian \citet{marchini2014quartet}, diberikan sistem yang mampu memprediksi tingkat keras suara, kecepatan bow, jangkauan \textit{vibrato}, dan perpanjangan not dari partitur untuk kuartet gesek. Dalam penelitian \citet{yu2017bowing}, diberikan sistem yang mampu memprediksi posisi bow untuk tiap not seperenambelasan dari partitur.

Riset terkait komponen CSEMP alat musik gesek lainnya hanya memberikan komponen sintesis suara alat musik. Synful dengan teknik RPM \parencite{lindemann2007rpm} mampu memberikan suara alat musik apabila diberikan sekuens not dengan gestur ekspresifnya. Namun, gestur ekspresif yang diberikan oleh riset komponen perencana gestur ekspresif alat musik gesek seperti \parencite{marchini2014quartet} dan \parencite{yu2017bowing} tidak dapat langsung menjadi masukan untuk Synful. Hal ini karena Synful didesain untuk menerima masukan dari pemain \textit{keyboard}. Gestur ekspresif pemain \textit{keyboard} berbeda dengan gestur ekspresif pemain alat musik gesek. Gestur dinamika \textit{keyboard} bersifat diskrit tiap not sedangkan gestur dinamika alat musik bersifat kontinu dengan dinamika dalam satu not dapat berubah. \textit{Keyboard} juga tidak memiliki gestur ekspresi \textit{timbre}, sedangkan pemain alat musik gesek dapat mengubah \textit{timbre} sebagai gestur ekspresi.

Terdapat pula riset sintesis alat gesek hanya membahas sintesis satu not, seperti NSynth \parencite{nsynth2017}. Sebuah sistem lain yang diberikan oleh \citet{yang2016synthesis} mampu membangkitkan suara alat musik gesek hanya dari partitur dan sebutan ekspresi musik, namun sintesis suara tiap not-nya tidak memperhatikan konteks not dalam karya musik. Suara yang dihasilkan sistem ini tidak akan memiliki variasi tempo, variasi dinamika antar not, artikulasi, dan gestur-gestur ekspresi lainnya yang bergantung kepada konteks not.

Terdapat pula framework untuk membangun CSEMP alat musik gesek yang mampu menghasilkan suara ekspresif dari partitur non-ekspresif. \parencite{perez2015} Implementasinya terhambat anotasi gestur pada data latih secara manual. Dalam permainan musik ekspresif, anotasi gestur secara manual sangat rentan kesalahan dan membutuhkan usaha sangat besar.

Begitu pula apabila hendak digunakan perencana gestur ekspresif yang telah ada \parencite{marchini2014quartet}\parencite{yu2017bowing} dan dibangun pensintesis suara baru yang menerima gestur tersebut, ataupun dibuat perencana gestur untuk masukan sistem pensintesis seperti sistem \citet{lindemann2007rpm}, akan dibutuhkan anotasi gestur pada data latih secara manual. Cara seperti ini rentan dengan kesalahan anotasi dan membutuhkan usaha sangat besar.

Karenanya, perlu dibangun sistem yang mampu menghasilkan suara ekspresif dari partitur non-ekspresif, tanpa perlu anotasi gestur secara manual dalam data latih. Teknik untuk sistem ini dapat diadopsi dari teknik yang telah digunakan dalam sistem pensintesis suara ekspresif untuk alat musik lainnya. Teknik dengan bantuan alat akustik yang digunakan dalam CSEMP piano yang disebutkan dalam \citet{schubert2017test} tidak dapat digunakan karena belum tersedia perangkat keras alat gesek akustik yang mampu menerima gestur ekspresif dari komputer.

%TODO perbaiki, hapus parametrik statistikal dan konkatenatif
Pilihan lainnya adalah menggunakan teknik dari sintesis suara nyanyian. Teknik suara nyanyian yang dapat diadopsi adalah teknik neural parametrik. Teknik ini mengatasi kelemahan-kelemahan yang ada pada teknik konkatenatif dan teknik parametrik statistikal\parencite{bonada2017singing}. Teknik-teknik konkatenatif \parencite{Bonada2016ExpressiveSS}\parencite{Bonada2007SynthesisOT} biasanya tidak fleksibel, dan memaksa timbre dilatih terpisah dari ekspresi \parencite{bonada2017singing}. Padahal, pada musik, khususnya alat musik dengan timbre bervariasi seperti vokal atau alat musik gesek, variasi timbre merupakan bagian dari parameter ekspresi. Teknik-teknik parametrik statistikal lebih fleksibel, namun tidak dapat mencapai kualitas suara seperti teknik konkatenatif.

Teknik sintesis untuk nyanyian ini dapat diadopsi kepada alat musik gesek karena keduanya memiliki kesamaan. Kesamaan alat musik gesek dan nyanyian adalah keduanya bersifat eksitasi-kontinu/\textit{sustained}. Berbeda dengan alat musik eksitasi-spontan seperti piano, penentuan variasi suara alat musik eksitasi-kontinu dilakukan secara terus-menerus sepanjang pembangkitan sinyal suara \parencite{Maestre2010StatisticalMO}. Selain itu, pada sebagian besar waktu, suatu sistem alat musik gesek hanya memainkan satu not, walaupun pada kasus yang sangat jarang alat musik gesek mungkin memainkan dua not dalam satu waktu. Nyanyian memiliki sifat yang mirip tersebut yaitu hanya memainkan satu not dalam satu waktu. Yang membedakan antara suara nyanyian dan alat musik gesek adalah tidak adanya fonem dalam sintesis alat musik gesek. Atas dasar kemiripan-kemiripan ini, adopsi teknik nyanyian kepada alat musik gesek menjadi lebih mungkin dilakukan daripada menggunakan komponen, dan \textit{framework} telah ada untuk alat musik gesek, dengan kelemahan-kelemahan yang telah disebutkan pada beberapa paragraf sebelumnya.

\section{Rumusan Masalah}

Untuk sebagian genre musik, dibutuhkan pensintesis yang mampu menghasilkan suara ekspresif. Pensintesis suara dengan taraf kealamian yang lebih tinggi lebih diharapkan. Untuk alat musik gesek, belum ada CSEMP yang mampu menghasilkan suara ekspresif dari partitur non-ekspresif saja. Hal ini karena komponen-komponen CSEMP alat musik gesek tidak kompatibel untuk digabungkan satu sama lain, ataupun karena framework CSEMP utuh yang ada terhambat masalah anotasi data. Teknik sintesis neural parametrik untuk nyanyian telah mampu menghasilkan suara dari partitur non-ekspresif saja, dan suara nyanyian memiliki beberapa kemiripan dengan alat musik gesek yang memungkinkan adopsi tekniknya kepada sistem untuk sintesis dan permainan alat musik gesek.

Perlu dibangun sebuah CSEMP alat musik gesek yang mampu menghasilkan suara ekspresif dari partitur non-ekspresif saja. Untuk membangun sistem tersebut, perlu dilakukan modifikasi dari teknik sintesis neural parametrik suara nyanyian untuk menyesuaikan dengan domain alat musik gesek.

\section{Tujuan}

Penelitian ini bertujuan untuk mengembangkan sebuah sistem komputer untuk permainan ekspresif alat musik gesek yang mampu menghasilkan suara ekspresif dari partitur non ekspresif dan meningkatkan kealamiannya. Untuk itu, dalam penelitian ini:
\begin{enumerate}
	\item akan dibangun sebuah sistem permainan ekspresif alat musik gesek partitur-ke-suara dengan teknik sintesis neural parametrik.
	\item akan dilakukan penyempurnaan terhadap teknik yang digunakan dalam sistem permainan ekspresif alat musik gesek dalam rangka meningkatkan kealamiannya.
\end{enumerate}

\section{Batasan Masalah}

\begin{enumerate}
	\item Sistem hanya memainkan satu instrumen, dengan satu not yang berbunyi pada satu waktu (monofonik)
	\item Alat musik gesek yang dimaksud adalah violin
	\item Kinerja CSEMP yang diukur dan diuji hanya koefisien korelasi terhadap referensi dan kealamian subjektif pendengar
\end{enumerate}

\section{Metodologi} \label{methodology}

Dalam penelitian ini, akan dilakukan langkah-langkah berikut ini:
\begin{enumerate}
	\item Menganalisis kekurangan dan batasan sistem-sistem permainan musik ekspresif neural parametrik yang sudah ada, serta perancangan solusi berupa modifikasi dan penyesuaian teknik untuk sistem permainan musik ekspresif alat musik gesek.
	\item Implementasi sistem permainan musik ekspresif alat musik gesek, yang meliputi implementasi program, pengumpulan data, dan pelatihan model
	\item Melakukan pengujian korelasi referensi dan kealamian.
\end{enumerate}

\section{Hipotesis}

Dengan mempertimbangkan bahwa teknik neural parametrik telah berhasil membangkitkan suara nyanyian ekspresif serupa alami dari partitur non-ekspresif saja, maka hipotesis penelitian ini adalah bahwa teknik neural parametrik dapat membangkitkan suara ekspresif untuk alat musik gesek dari partitur non ekspresif saja. Adopsi teknik sintesis neural parametrik nyanyian kepada sintesis alat musik gesek dapat dilakukan dengan menghilangkan aspek fonem dari teknik sintesis tersebut.

\section{Jadwal Pelaksanaan Penelitian}

Jadwal pelaksanaan penelitian ini tertuang dalam tabel \ref{tab-schedule}. Ketiga tahapan penelitian dalam subbab \ref{methodology} dilakukan secara berurutan. Untuk analisis dan perancangan solusi, dibutuhkan waktu 2 bulan.  Implementasi sistem terdiri dari implementasi program, pengumpulan data, dan pelatihan model, membutuhkan waktu total 3,5 bulan. Implementasi program membutuhkan waktu 3 bulan dan dapat dilakukan ketika sebagian besar analisis dan perancangan solusi telah selesai dilakukan. Pengumpulan data membutuhkan waktu 2 bulan. Pengumpulan data dapat dilakukan setelah analisis dan perancangan solusi selesai, paralel dengan implementasi program. Pelatihan model membutuhkan waktu 2 minggu, dan dapat dilakukan setelah program dan data tersedia. Hasil pelatihan model kemudian digunakan untuk pengujian. Pengujian dan analisis hasil akan membutuhkan waktu 1 bulan.

\begin{table}[h]
	\centering
    \caption{Jadwal Pelaksanaan Penelitian}\label{tab-schedule}
	\begin{tabular}{ |c|l|p{5px}|p{5px}|p{5px}|p{5px}|p{5px}|p{5px}|p{5px}|p{5px}|p{5px}|p{5px}|p{5px}|p{5px}| } 
	\hline
	No & Tahapan & \multicolumn{2}{p{10px}|}{Ag 2018} & \multicolumn{2}{p{10px}|}{Sep 2018} & \multicolumn{2}{p{10px}|}{Okt 2018} & \multicolumn{2}{p{10px}|}{Nov 2018} & \multicolumn{2}{p{10px}|}{Des 2018}  & \multicolumn{2}{p{10px}|}{Jan 2019} \\
	\hline
	1 & Analisis dan Perancangan Solusi & \fillcell & \fillcell & \fillcell & \fillcell & & & & & & & & \\
	\hline
	2 & Implementasi Sistem& & & & \fillcell & \fillcell &\fillcell &\fillcell &\fillcell &\fillcell &\fillcell & & \\
	\hline
	2a & \quad Implementasi Program & & & & \fillcell & \fillcell &\fillcell &\fillcell &\fillcell &\fillcell & & & \\
	\hline
	2b & \quad Pengumpulan Data & & & & & \fillcell &\fillcell &\fillcell &\fillcell & & & & \\
	\hline
	2c & \quad Pelatihan Model & & & & &  & & & & &\fillcell & & \\
	\hline
	3 & Pengujian dan Analisis Hasil & & & & &  & & & & & &\fillcell &\fillcell \\
	\hline
	\end{tabular}
	
\end{table}