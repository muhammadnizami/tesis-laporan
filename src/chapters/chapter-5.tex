\chapter{Penutup}

\section{Kesimpulan}

% masalah:Untuk sebagian genre musik, dibutuhkan pensintesis yang mampu menghasilkan suara ekspresif. Pensintesis suara dengan taraf kealamian yang lebih tinggi lebih diharapkan. Untuk alat musik gesek, belum ada CSEMP yang mampu menghasilkan suara ekspresif dari partitur non-ekspresif saja. Hal ini karena komponen-komponen CSEMP alat musik gesek tidak kompatibel untuk digabungkan satu sama lain, ataupun karena \textit{framework} CSEMP utuh yang ada terhambat masalah anotasi data. Teknik sintesis neural parametrik untuk nyanyian telah mampu menghasilkan suara dari partitur non-ekspresif saja, dan suara nyanyian memiliki beberapa kemiripan dengan alat musik gesek yang memungkinkan adopsi tekniknya kepada sistem untuk sintesis dan permainan alat musik gesek.

% Perlu dibangun sebuah CSEMP alat musik gesek yang mampu menghasilkan suara ekspresif dari partitur non-ekspresif saja. Untuk membangun sistem tersebut, perlu dilakukan modifikasi dari teknik sintesis neural parametrik suara nyanyian untuk menyesuaikan dengan domain alat musik gesek.

% tujuan penelitian: Penelitian ini bertujuan untuk mengembangkan sebuah sistem komputer untuk permainan ekspresif alat musik gesek yang mampu menghasilkan suara ekspresif dari partitur non ekspresif. Untuk itu, dalam penelitian ini akan dibangun sebuah sistem permainan ekspresif alat musik gesek partitur-ke-suara dengan teknik sintesis neural parametrik.

Dicapai kesimpulan sebagai berikut:

\begin{enumerate}

\item Telah dibangun sebuah sistem komputer untuk permainan ekspresif alat musik gesek yang mampu menghasilkan suara ekspresif dari partitur non-ekspresif. Sistem permainan ekspresif alat musik gesek partitur-ke-suara dibangun dengan teknik sintesis neural parametrik. Modifikasi yang digunakan terdapat pada pengkodean yang digunakan, komponen model \textit{timing} dan \textit{timbre}, arsitektur model \textit{timing}, serta fitur-fitur masukan untuk model-model \textit{timing}, \textit{pitch}, dan \textit{timbre}.

\item Penggunaan \textit{decision tree} untuk model \textit{timing} berhasil menghasilkan \textit{timing} yang memiliki kinerja lebih baik daripada model \textit{timing} yang menggunakan jaringan syaraf tiruan, dengan metrik kinerja koefisien korelasi nada yang bersesuaian.

\item Uji korelasi sistem ini menunjukkan bahwa sistem ini memiliki kinerja -dengan metrik koefisien korelasi- lebih baik daripada sistem pensintesis \textit{baseline} yaitu RPM tanpa masukan gestur ekspresi. Koefisien korelasi sistem ini bernilai 0.1133, sedangkan RPM memiliki koefisien korelasi -0.0020.

\item Kealamian subjektif pendengar sistem permainan musik ekspresif alat musik gesek dengan teknik neural parametrik lebih rendah daripada sistem RPM. Nilai preferensi sistem permainan musik alat musik gesek dengan teknik neural parametrik terhadap RPM adalah $19,44\%$. %TODO analisis
Sistem permainan musik ekspresif nyanyian dengan teknik gabungan gestur ekspresi neural parametrik dan pensintesis RPM menghasilkan kealamian subjektif pendengar yang lebih rendah daripada sistem RPM. Nilai preferensi sistem permainan musik alat musik gesek dengan teknik neural parametrik terhadap RPM adalah $31,25\%$. %TODO analisis
Nilai preferensi untuk kealamian subjektif pendengar ini bertolak belakang dengan hasil uji korelasi sistem. Dengan uji korelasi, teknik neural parametrik memiliki kinerja lebih tinggi daripada RPM, sedangkan dengan uji preferensi kealamian, teknik neural parametrik memiliki kinerja lebih rendah daripada RPM.

\item Gestur-gestur ekspresi seperti \textit{timing}, deviasi \textit{pitch}, dan variasi dinamika dari sistem yang diajukan sudah dianggap alami. Meski demikian, \textit{timbre} yang belum mirip dengan suara alat musik asli dan juga tidak ikut berubah mengikuti gestur-gestur ekspresi tersebut menyebabkan keluaran sistem neural parametrik secara keseluruhan dianggap kurang alami.

\item Jaringan syaraf tiruan dengan arsitektur WaveNet yang dimodifikasi mampu menghasilkan pola-pola jangka panjang namun pola-pola detil jangka pendek tidak mampu ia pelajari. Karena inilah gestur ekspresi deviasi \textit{timing}, deviasi \textit{pitch}, dan variasi dinamika menjadi lebih alami sementara \textit{timbre} yang dihasilkan tidak alami. Jaringan syaraf tiruan dengan arsitektur WaveNet yang dimodifikasi tidak cocok dengan data kode harmonik plus stokastik alat musik gesek yang tidak mulus dan memiliki banyak pola perubahan jangka-pendek.

\end{enumerate}

\section{Saran}

Dari penelitian ini, terdapat beberapa saran baik untuk memperbaiki sistem yang digunakan maupun validitas hasil penelitian-penelitian serupa:
\begin{enumerate}
	\item Peningkatan kealamian sistem dapat dilakukan apabila dibuat arsitektur model yang mampu menangkap pola lebih kompleks dan lebih detil agar ekspresi terkait \textit{pitch} dan \textit{timbre} dari data latih benar-benar dapat direproduksi oleh model. Perbaikan arsitektur model ini lebih penting daripada penambahan data latih karena penambahan data latih tanpa perbaikan arsitektur model \textit{pitch} dan \textit{timbre} justru akan menurunkan kinerja.
	\item Perbaikan arsitektur model harus mempertimbangkan kebutuhan waktu eksekusi dan memori
	\item Peningkatan kealamian sistem juga dapat dilakukan dengan menggunakan, bila ada, pengkodean suara yang lebih mulus dan tidak memiliki pola-pola jangka-pendek yang terlalu kompleks.
	\item Alternatif lain dari perbaikan sistem adalah tetap menggunakan arsitektur yang diajukan untuk menangkap gestur-gestur ekspresi deviasi \textit{timing}, deviasi \textit{pitch} dan variasi dinamika namun ditambah model lain khusus untuk menangkap pola \textit{timbre}.
	\item Perlu dibuat metrik yang mampu menilai kemiripan bentuk variasi jangka-pendek dari keluaran model-model. Variasi ini meliputi \textit{spike}, osilasi, saling silang antar fitur dan sebagainya.
	\item Penelitian dapat dilanjutkan ke aspek \textit{correctness} dari sistem, terutama untuk gestur-gestur deviasi \textit{timing}, deviasi \textit{pitch}, dan dinamika.
	\item Perlu dilakukan perluasan \textit{scope} dari sistem agar sistem mampu menerima partitur-partitur selain partitur monofonik.
	\item Teknik yang digunakan juga dapat diterapkan ke alat-alat musik gesek selain violin.
\end{enumerate}