\chapter*{Kata Pengantar}
\addcontentsline{toc}{chapter}{Kata Pengantar}

Puji dan syukur penulis ucapkan kepada Tuhan Yang Maha Esa, karena atas rahmat dan karunia-Nya penulis dapat menyelesaikan penulisan laporan tesis yang berjudul: "Sistem Permainan Musik Ekspresif untuk Alat Musik Gesek". Laporan ini dibuat sebagai syarat kelulusan pada Program Studi Magister Informatika, Institut Teknologi Bandung. Penyelesaian tugas akhir ini, tentu tidak terlepas dari dukungan berbagai pihak. Oleh karena itu, penulis ucapkan terima kasih yang sebesar-besarnya kepada:

\begin{enumerate}
\item Ibu Dessi Puji Lestari S.T.,M.Eng.,Ph.D., selaku dosen pembimbing, atas ilmu, arahan, saran, motivasi dan kesabaran dalam membimbing penulis selama proses tesis
\item Ibu Dr. Nur Ulfa Maulidevi, S.T., M.Sc selaku dosen penguji, atas kritik, masukan dan saran terkait pengerjaan tesis. %TODO tambah dengan penguji sidang
% 
\item %Dr. Fazat Nur Azizah S.T., M.Sc. dan Adi Mulyanto S.T., M.T. selaku
Dosen koordinator tesis, beserta seluruh tim tesis, atas bimbingan dan arahan dalam memudahkan proses administrasi kuliah tugas akhir,
\item Ibu Dr. Masayu Leylia Khodra ST,MT., selaku dosen wali, atas arahan dan dukungan selama penulis menempuh pendidikan di Program Studi Magister Informatika ITB,
\item Dr. Bayu Hendradjaya, ST.,MT., selaku Ketua Program Studi Magister Informatika ITB, atas segala perhatian dan dukungan kepada seluruh mahasiswa di Program Studi Magister Informatika ITB,
\item Dosen, staf pengajar, asisten akademik, asisten lab, staf program studi, staf STEI, tenaga kebersihan, tenaga keamanan, sarpras, dan civitas ITB lainnya yang turut mendukung studi penulis di Program Studi Teknik Informatika ITB,
\item Orang tua dan keluarga penulis yang selalu mendoakan, mendukung, dan menyemangati penulis dalam menjalankan tesis,
\item Rekan-rekan mahasiswa Program Studi Magister Informatika ITB dan Program Studi Teknik Informatika ITB yang selalu memberikan dukungan, semangat, dan bantuan ilmu selama berkuliah di Institut Teknologi Bandung,
\item Teman-teman musisi yang telah membantu proses pengujian dan memberikan umpan balik,
\item Seluruh pihak yang berkontribusi besar dalam memberikan dukungan bagi penulis, namun tidak dapat disebutkan satu per satu. Semoga tugas akhir ini dapat memberikan manfaat dan kontribusi dalam bidang ilmu informatika khususnya pada riset kecerdasan buatan, pengolahan suara, dan lebih khusus lagi pada bidang suara musik.
\end{enumerate}

Akhir kata, penulis ucapkan terima kasih.
Bandung, Mei 2019
Penulis