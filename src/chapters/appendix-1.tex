\chapter{Pendapat-Pendapat Responden}\label{appendix-pendapat}

	\begin{longtable}{|p{0.6\textwidth}|r|r|r|r|r|}
	\hline
Pendapat&	Total&	\multicolumn{4}{|c|}{Pasangan, Teknik yang Dipilih}\\
\cline{3-6}
&	&	\multicolumn{2}{|c|}{DTNP/RPM}	&\multicolumn{2}{|c|}{Gab./RPM}\\
\cline{3-6}
&	&	RPM&	DT-NP&	RPM&	Gab.\\\hline
\endhead
\begin{intabquote}variasi warna suara\end{intabquote}&	168&	68&	11&	59&	30\\\hline
\begin{intabquote}audio yang tidak dipilih seperti audio yang sengaja didistorsi\end{intabquote}&	151&	64&	10&	59&	18\\\hline
\begin{intabquote}variasi dinamika\end{intabquote}&	130&	44&	13&	45&	28\\\hline
\begin{intabquote}variasi timing\end{intabquote}&	119&	31&	23&	38&	27\\\hline
\begin{intabquote}deviasi pitch\end{intabquote}&	104&	33&	13&	36&	22\\\hline
\begin{intabquote}seperti audio yang sengaja di-bend ekualisasinya\end{intabquote}&	96&	44&	4&	39&	9\\\hline
\begin{intabquote}vibrato\end{intabquote}&	88&	32&	9&	29&	18\\\hline
\begin{intabquote}phrasing\end{intabquote}&	87&	37&	4&	37&	9\\\hline
\begin{intabquote}keduanya tidak terdengar natural\end{intabquote}&	76&	28&	9&	26&	13\\\hline
\begin{intabquote}pitch jelas\end{intabquote}&	67&	32&	1&	28&	6\\\hline
\begin{intabquote}cara dinamikanya beda2\end{intabquote}&	62&	17&	8&	16&	21\\\hline
\begin{intabquote}karena phrase tersebut terdengar lebih natural dengan timbre dari media yang digunakan pada audio yang saya pilih\end{intabquote}&	56&	21&	6&	20&	9\\\hline
\begin{intabquote}suara jelas\end{intabquote}&	54&	25&	1&	27&	1\\\hline
\begin{intabquote}tidak ada perubahan dinamika mendadak\end{intabquote}&	54&	25&	1&	27&	1\\\hline
\begin{intabquote}audio yang tidak dipilih seperti di-mixing pada aplikasi dengan timing yang salah\end{intabquote}&	51&	25&	0&	24&	2\\\hline
\begin{intabquote}ada sedikit crescendo\end{intabquote}&	38&	10&	6&	12&	10\\\hline
\begin{intabquote}Audio yang tidak dipilih terdengar noisy\end{intabquote}&	35&	16&	1&	14&	4\\\hline
\begin{intabquote}fingering nadanya dan cara stroke-nya terdengar lebih manusiawi, walaupun terdengar seperti orang yang bermain asal dan terburu-buru\end{intabquote}&	32&	5&	8&	8&	11\\\hline
\begin{intabquote}trill-nya terdengar aneh tapi overall play-nya terdengar lebih natural\end{intabquote}&	30&	6&	6&	8&	10\\\hline
\begin{intabquote}fingering seperti sedang meraih-raih nada\end{intabquote}&	30&	4&	8&	8&	10\\\hline
\begin{intabquote}permainan long bow-nya terasa alami\end{intabquote}&	30&	9&	3&	10&	8\\\hline
\begin{intabquote}phrasing-nya mungkin terasa seperti "terlalu tepat", tapi jadinya lebih terdengar seperti pemusik yang bermain pada ketukan cepat dengan timing yang tepat\end{intabquote}&	28&	11&	3&	13&	1\\\hline
\begin{intabquote}pada audio yang tidak dipilih, perpindahan pitch dan "stroke"-nya seperti dimainkan pada aplikasi music sheet, tidak berekspresi sama sekali\end{intabquote}&	25&	6&	4&	6&	9\\\hline
\begin{intabquote}pada audio yang tidak dipilih, terasa seperti decressendo, padahal lebih terdengar seperti gain-nya saja dikurangi\end{intabquote}&	22&	10&	0&	11&	1\\\hline
\begin{intabquote}walaupun di audio yang tidak dipilih terdengar seperti ada crossing senar\end{intabquote}&	19&	6&	1&	9&	3\\\hline
\begin{intabquote}adanya rubato yang sebenarnya memberikan sedikit ekspresi dibandingkan audio yang satunya, tapi entah kenapa masih ada yang mengganjal\end{intabquote}&	16&	2&	3&	3&	8\\\hline
\begin{intabquote}sebenernya saya lebih suka timbre dari audio yang tidak dipilih (untuk phrasing ini), tapi mainnya gak rapih jadi terganggu waktu mau menikmati suaranya\end{intabquote}&	15&	2&	5&	2&	6\\\hline
\begin{intabquote}bingung pisan denger apa yang salah satunya teh ?!?\end{intabquote}&	11&	5&	1&	4&	1\\\hline
\begin{intabquote}apa ini....\end{intabquote}&	2&	1&	0&	1&	0\\\hline
\begin{intabquote}sebenernya ga kedengeran kaya cross string, tapi yang ini lebih mending dari yang satu lagi\end{intabquote}&	2&	1&	0&	1&	0\\\hline
\begin{intabquote}ini kayanya audio yang gue pilih sama persis kaya audio sebelumnya (?) hahahaha\end{intabquote}&	2&	1&	0&	1&	0\\\hline
\begin{intabquote}suaranya kaya lebih natural yang ini tapi di-bend (?)\end{intabquote}&	2&	0&	0&	2&	0\\\hline
\begin{intabquote}ini juga... hahahaha. suaranya kaya lebih natural tapi ga tau diapain sama kamu jadi sucks HAHAHA\end{intabquote}&	1&	0&	0&	1&	0\\\hline

	\end{longtable}