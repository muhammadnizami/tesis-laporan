\clearpage
\chapter*{Abstract}
\addcontentsline{toc}{chapter}{Abstract}
\begin{center}
	\singlespacing
    \large \bfseries \MakeUppercase{Guidelines for Informatics Final Project} %\thetitle translated into english

    \normalsize \normalfont Oleh

    \bfseries \large \theauthor\\
    \normalsize (Program Studi Magister Informatika)
    \bigskip
\end{center}

\begin{singlespace}

%put your abstract here
Expressive performance is more preferrable than non-expressive ones in some genres such as classical music. Previously, there has been developed sheet-to-sound CSEMPs (Computer System for Expressive Musical Performance) for some instruments, such as piano and singing voice. There is not yet any CSEMP for string instrument that can produce expressive sound just from non-expressive musical sheet. Integration of existing strings CSEMP components cannot be done because of incompatibility of each component's input and output representations. In this work, a new system for synthesis and performance of string instruments is constructed, adopting neural parametric synthesis, a technique that was used previously for singing synthesis.

From this neural parametric technique, some adjustments was done. Input data adjustment is performed by removing the need of inputs of phonetic aspects. The coding is replaced with harmonic plus stochastic model, which is more suitable for string instruments. For timing model, decision tree is used instead of a neural network because neural network only generates constant timing deviations. For pitch and timbre model, the same modified WaveNet is used. Musical sheet features are added to the input features for timbre model. Sparse-dense dilations for causal convolutions is used in place of the original dilated causal convolutions. These models are trained with a training set of sheet-recording pairs.

The experiment and testing is composed of three steps, i.e. parameter tuning, reference correlation test, and listener's naturalness perception preference. Correlation coefficient value of this system against test data is 0.1133, which is better than baseline's -0.0020 value. However, the produced sounds are preceived as less natural than RPM. Listener's naturalness preference score of this system against RPM is $19,44\%$.

\end{singlespace}
\clearpage