\clearpage
\chapter*{Abstract}
\addcontentsline{toc}{chapter}{Abstract}

%put your abstract here
Despite non-expressive synthesis and performance being more popular for synth-pop and dance music, expressive performance is more preferrable in other genres such as classical music. Computer systems able to perform musical instruments expressively are known as Computer Systems for Expressive Musical Performance (CSEMP). Previously, there has been developed CSEMPs for some instruments, such as piano and singing voice, that can generate expressive audio from non-expressive musical sheet. There is not yet any CSEMP for string instrument that can produce expressive sound just from non-expressive musical sheet. Integration of existing strings CSEMP components cannot be done because of incompatibility of each component's input and output representations. In this work, a new system for synthesis and performance of string instruments will be constructed, adopting neural parametric synthesis, a technique that was used previously for singing synthesis. In a previous work, with separate timing, pitch, and timbre models, neural parametric synthesis technique was able to quickly generate expressive singing voice from non-expressive musical sheets.

From this neural parametric technique, some adjustments and improvements will be done. Input data adjustment is performed by removing the need of inputs of phonetic aspects. Improvements so that the synthesizer recognizes contextual patterns for timbre and pitch decisions is performed by modifying the input features and modifying the neural network architecture. Musical sheet features are added to the input features for timbre model. Sparse-dense dilations for causal convolutions is used in place of the original dilated causal convolutions. These models will be trained with a training set of sheet-recording pairs.

\clearpage