\clearpage
\chapter*{ABSTRAK}
\addcontentsline{toc}{chapter}{Abstrak}
\begin{center}
	\singlespacing
    \large \bfseries \MakeUppercase{\thetitle}

    \normalsize \normalfont Oleh

    \bfseries \large \theauthor\\
    \normalsize (Program Studi Magister Informatika)
    \bigskip
\end{center}

\begin{singlespace}

%taruh abstrak bahasa indonesia di sini
Permainan ekspresif lebih disukai daripada non-ekspresif untuk beberapa genre seperti musik klasik. Sebelumnya, telah dikembangkan CSEMP (\textit{Computer System for Expressive Musical Performance} - Sistem Komputer untuk Permainan Musik Ekspresif) partitur-ke-suara untuk beberapa alat musik, misalnya piano dan vokal. Belum ada CSEMP alat musik gesek yang mampu menghasilkan suara ekspresif dari partitur non-ekspresif. Penggabungan komponen-komponen CSEMP yang tersedia terkendala inkompatibilitas input dan output antar komponen. Untuk itu, dibangun sebuah sistem permainan dan sintesis alat musik gesek baru yang mengadopsi teknik sintesis neural parametrik yang telah teruji pada sintesis nyanyian.

Dari teknik neural parametrik ini, dilakukan penyesuaian dan perbaikan. Penyesuaian data masukan dilakukan dengan menghapus kebutuhan masukan aspek fonetik. Pengkodean diganti dengan pengkodean harmonik plus stokastik yang lebih sesuai untuk alat musik gesek. Karena model \textit{timing} dengan jaringan syaraf tiruan hanya menghasilkan deviasi \textit{timing} konstan, modle \textit{timing} diganti menjadi menggunakan \textit{decision tree}. Model \textit{pitch} dan \textit{timbre} tetap menggunakan jaringan syaraf tiruan Wavenet yang dimodifikasi. Fitur masukan model \textit{timbre} ditambah dengan fitur dari partitur. Model-model dilatih dengan data latih kumpulan pasangan partitur-rekaman.

Eksperimen dan pengujian terdiri dari tiga tahap, yaitu penyetelan parameter, uji korelasi referensi, dan uji preferensi kealamian pendengar. Nilai koefisien korelasi sistem ini terhadap data uji adalah 0.1133, lebih baik dari \textit{baseline} RPM yang memiliki nilai koefisien korelasi -0.0020. Namun, oleh pendengar, suara yang dihasilkan masih dipersepsi kurang alami daripada RPM. Nilai preferensi kealamian pendengar sistem ini terhadap RPM adalah $19,44\%$.


\end{singlespace}
\clearpage