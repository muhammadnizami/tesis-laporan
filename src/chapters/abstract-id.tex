\clearpage
\chapter*{ABSTRAK}
\addcontentsline{toc}{chapter}{Abstrak}

%taruh abstrak bahasa indonesia di sini
Meski permainan dan sintesis non-ekspresif yang kaku lebih populer untuk ragam \textit{synth-pop} dan \textit{dance music}, untuk genre lain seperti musik klasik permainan yang ekspresif lebih disukai. Sistem komputer yang mampu memainkan alat musik secara ekspresif disebut sebagai sistem komputer untuk permainan musik ekspresif (\textit{Computer Systems for Expressive Musical Performance} - CSEMP). Sebelumnya, telah dikembangkan CSEMP untuk beberapa alat musik, misalnya piano dan vokal, yang mampu membangkitkan suara ekspresif dari partitur non-ekspresif. Belum ada CSEMP alat musik gesek yang mampu menghasilkan suara ekspresif dari partitur non-ekspresif. Penggabungan komponen-komponen CSEMP alat musik gesek yang tersedia tidak dapat dilakukan karena input dan output tiap komponen tidak kompatibel satu sama lain. Untuk itu, akan dibangun sebuah sistem permainan dan sintesis alat musik gesek baru yang mengadopsi teknik sintesis parametrik neural yang telah teruji pada sintesis nyanyian. Dalam riset sebelumnya, dengan model-model \textit{timing}, \textit{pitch}, dan \textit{timbre} yang terpisah, teknik sintesis neural parametrik mampu menghasilkan suara nyanyian ekspresif dari partitur non-ekspresif saja dengan cepat.

Dari teknik parametrik neural ini, akan dilakukan penyesuaian dan perbaikan. Penyesuaian data masukan dilakukan dengan menghapus kebutuhan masukan aspek fonetik. Perbaikan agar pensintesis mampu menangkap pola kontekstual untuk pemilihan \textit{timbre} dan \textit{pitch} dilakukan dengan penambahan fitur masukan dan perubahan arsitektur jaringan syaraf tiruan yang digunakan. Fitur masukan model \textit{timbre} ditambah dengan fitur dari partitur. Dilasi konvolusi kausal pada jaringan syaraf tiruan yang diubah menjadi dilasi konvolusi kausal renggang-padat. Model-model akan dilatih dengan data latih kumpulan pasangan partitur-rekaman.

\clearpage