\newglossaryentry{partitur}
{
    name=partitur,
    description={bentuk tertulis komposisi musik}
}
\newglossaryentry{tempo}
{
    name=tempo,
    description={kecepatan dari ritme sebuah komposisi, diukur dalam ketukan per menit}
}
\newglossaryentry{dinamika}
{
	name=dinamika,
	description={tingkat keras atau lembut dari komposisi. Contoh istilah-istilah terkait dinamika adalah \textit{piano}, \textit{forte}, \textit{crescendo}, \textit{decrescendo}}
}
\newglossaryentry{piano_dinamika}
{
	name={\textit{piano} (dinamika)},
	description={dimainkan dengan lembut}
}
\newglossaryentry{forte}
{
	name=\textit{forte},
	description={dimainkan dengan keras}
}
\newglossaryentry{crescendo}
{
	name=\textit{crescendo},
	description={bertambah keras secara perlahan}
}
\newglossaryentry{decrescendocrescendo}
{
	name=\textit{decrescendo},
	description={bertambah lembut secara perlahan}
}
\newglossaryentry{artikulasi}
{
	name=artikulasi,
	description={arahan untuk suatu not, yang menunukkan karakteristik \textit{attack}, durasi, dan \textit{decay}}
}
\newglossaryentry{not}
{
	name=not,
	description={sebuah simbol notasional yang digunakan untuk merepresentasikan durasi dan \textit{pitch} dari suara}
}
\newglossaryentry{pitch}
{
	name=\textit{pitch},
	description={tinggi rendah suara}
}
\newglossaryentry{bow}
{
	name=\textit{bow},
	description={alat untuk menggesek senar}
}
\newglossaryentry{bowing}
{
	name=\textit{bowing},
	description={cara menggesek, aksi menggesek}
}
\newglossaryentry{vibrato}
{
	name=\textit{vibrato},
	description={Osilasi \textit{pitch} dalam satu not}
}
\newglossaryentry{ekspresi}
{
	name=ekspresi,
	description={Sentuhan-sentuhan dalam permainan musik seperti dinamika, tempo, dan artikulasi}
}
\newglossaryentry{waveform}
{
	name=\textit{waveform},
	description={bentuk sinyal gelombang, naik turunnya tegangan yang terekam}
}
\newglossaryentry{espressivo}
{
	name=\textit{espressivo},
	description={ekspresif}
}
\newglossaryentry{scherzando}{
	name=\textit{scherzando},
	description={melawak}
}
\newglossaryentry{tranquillo}
{
	name=\textit{tranquillo},
	description={tenteram}
}
\newglossaryentry{timbre}
{
	name=\textit{timbre},
	description={kualitas atau warna dari suara}
}
\newglossaryentry{CSEMP}
{
	name=\textit{CSEMP},
	description={\textit{Computer Systems for Expressive Musical Performance}, sistem komputer untuk permainan musik ekspresif, sistem komputer yang mampu memainkan alat musik secara ekspresif}
}
\newglossaryentry{monofonik}
{
	name=monofonik,
	description={hanya satu not yang berbunyi pada satu waktu}
}
\newglossaryentry{semitone}
{
	name=\textit{semitone},
	description={ukuran jarak dengan skala logaritmik terhadap frekuensi, jarak antara dua not yang bersebelahan dalam tangga nada kromatik, misalnya seperti antara C dan C\# atau C\# dan D, adalah satu \textit{semitone}}
}
\newglossaryentry{oktaf}
{
	name=\textit{oktaf},
	description={rentang not-not di antara dua not di mana not atas memiliki frekuensi dua kali lipat not bawah, atau berjarak 12 \textit{semitone}	}
}